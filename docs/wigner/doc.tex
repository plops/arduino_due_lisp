\documentclass{article}
\usepackage{amssymb}
\usepackage{amsmath}
\usepackage{amsbsy} % for bold greek
\newcommand{\vect}[1]{\mathbf{#1}}
\renewcommand{\r}{\vect r}
\newcommand{\p}{\vect p}
\renewcommand{\k}{\vect k}
\newcommand{\vnu}{\boldsymbol\nu}
\begin{document}
the following from Gross: ``Handbook of optical systems'' 2005 Wiley-VCH

definition of the coherence function $\Gamma$:
\begin{align}
  \Gamma_{12}(\tau) &= \frac{1}{T}\int_t^{t+T} U(\r_1,t+\tau) U^*(\r_2,t)\textrm{d}t
\end{align}

permuting the spatial vectors results in the complex conjugate

inserting two complex amplitudes:

\begin{align}
  \Gamma(U_n, U_m) &= \frac{1}{T}\int_0^T A_n A_m \cos\left[ \Delta k_{nm} r -\Delta\omega_{nm}(t+\tau) + \Delta\phi_{nm} \right]\textrm{d}t
\end{align}

for $n=m$ this expression is the intensity and is consequently called mutual intensity

complex degree of coherence $\gamma\in[0,1]$

\begin{align}
  \gamma_{12}(\tau) &= \frac{\Gamma_{12}(\tau)}{\sqrt{\Gamma_{11}(0)\Gamma_{22}(0)}} = \frac{\Gamma(\r_1,\r_2,\tau)}{\sqrt{I(\r_1)I(\r_2)}}
\end{align}

diese Funktion wird mit zunehmenden Abstand in Raum $\Delta\r$ und Zeit $\tau$ kleiner

Anwendung der Wellengleichung in Helmholtzformulierung f\"uhrt zu zwei
gekoppelten Wellengleichungen in beiden Ortskoordinaten f\"ur die
Koh\"arenzfunktion:
\begin{align}
  \nabla_j^2\,\Gamma - \frac{1}{c^2} \frac{\partial^2\, \Gamma}{\partial t^2} = 0, \quad j=1,2
\end{align}

Einf\"uhrung der Wignerdistribution:

Nutze Schwerpunkt und Differenzkoordinaten:
\begin{align}
  \r &= \frac{\r_1+\r_2}{2} \\
  \Delta\r &= \r_1-\r_2
\end{align}


\begin{align}
  W(\r,\vnu) &= \int\Gamma\left(\r+\frac{\Delta\r}{2},\r+\frac{\Delta\r}{2}\right) e^{-2\pi i \Delta\r\vnu}\textrm{d}\Delta\r
\end{align}

Eine oft genutzte alternative Formulierung ersetzt die transverse
Ortsfrequenz $\nu$ mit den $x-$ und $y-$Komponenten des optischen
Richtungskosinusvektors $\p$. Diese sind gegeben durch seinen Winkel
$u$ mit der optischen Achse:

\begin{align}
  \p &= \sin\vect u = \lambda \vnu = \frac{\lambda}{2\pi} \k
\end{align}

\begin{align}
  W'(\r,\p) &= \int\Gamma\left(\r+\frac{\Delta\r}{2},\r+\frac{\Delta\r}{2}\right) e^{-2\pi i \Delta\r\p}\textrm{d}\Delta\r
\end{align}


Die Koh\"arenzfunktion $\Gamma$ und die Wignerdistribution $W'$
enthalten die gleichen Informationen. Beide sind vierdimensional. $W'$
ist eine Quasidichteverteilung\footnote{Quasidichteverteilungen
  liefern zwar den Erwartungswert aber die zugrundeliegenden Gr\"osse
  ist nicht wie bei Wahrscheinlichkeitsverteilung unabh\"angig.}

\end{document}