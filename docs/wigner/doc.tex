\documentclass{article}
\usepackage{amssymb}
\usepackage{amsmath}
\usepackage{amsbsy} % for bold greek
\newcommand{\vect}[1]{\mathbf{#1}}
\renewcommand{\r}{\vect r}
\newcommand{\p}{\vect p}
\newcommand{\trmd}{\textrm{d}}
\renewcommand{\k}{\vect k}
\newcommand{\vnu}{\boldsymbol\nu}
\begin{document}
the following from Gross: ``Handbook of optical systems'' 2005 Wiley-VCH

definition of the coherence function $\Gamma$:
\begin{align}
  \Gamma_{12}(\tau) &= \frac{1}{T}\int_t^{t+T} U(\r_1,t+\tau) U^*(\r_2,t)\textrm{d}t
\end{align}

permuting the spatial vectors results in the complex conjugate

inserting two complex amplitudes:

\begin{align}
  \Gamma(U_n, U_m) &= \frac{1}{T}\int_0^T A_n A_m \cos\left[ \Delta k_{nm} r -\Delta\omega_{nm}(t+\tau) + \Delta\phi_{nm} \right]\textrm{d}t
\end{align}

for $n=m$ this expression is the intensity and is consequently called mutual intensity

complex degree of coherence $\gamma\in[0,1]$

\begin{align}
  \gamma_{12}(\tau) &= \frac{\Gamma_{12}(\tau)}{\sqrt{\Gamma_{11}(0)\Gamma_{22}(0)}} = \frac{\Gamma(\r_1,\r_2,\tau)}{\sqrt{I(\r_1)I(\r_2)}}
\end{align}

diese Funktion wird mit zunehmenden Abstand in Raum $\Delta\r$ und Zeit $\tau$ kleiner

Anwendung der Wellengleichung in Helmholtzformulierung f\"uhrt zu zwei
gekoppelten Wellengleichungen in beiden Ortskoordinaten f\"ur die
Koh\"arenzfunktion:
\begin{align}
  \nabla_j^2\,\Gamma - \frac{1}{c^2} \frac{\partial^2\, \Gamma}{\partial t^2} = 0, \quad j=1,2
\end{align}

Einf\"uhrung der Wignerdistribution:

Nutze Schwerpunkt und Differenzkoordinaten:
\begin{align}
  \r &= \frac{\r_1+\r_2}{2} \\
  \Delta\r &= \r_1-\r_2
\end{align}


\begin{align}
  W(\r,\vnu) &= \int\Gamma\left(\r+\frac{\Delta\r}{2},\r+\frac{\Delta\r}{2}\right) e^{-2\pi i \Delta\r\vnu}\textrm{d}\Delta\r
\end{align}

Eine oft genutzte alternative Formulierung ersetzt die transverse
Ortsfrequenz $\nu$ mit den $x-$ und $y-$Komponenten des optischen
Richtungskosinusvektors $\p$. Diese sind gegeben durch seinen Winkel
$u$ mit der optischen Achse:

\begin{align}
  \p &= \sin\vect u = \lambda \vnu = \frac{\lambda}{2\pi} \k
\end{align}

\begin{align}
  W'(\r,\p) &= \int\Gamma\left(\r+\frac{\Delta\r}{2},\r+\frac{\Delta\r}{2}\right) e^{-2\pi i \Delta\r\p}\textrm{d}\Delta\r
\end{align}


Die Koh\"arenzfunktion $\Gamma$ und die Wignerdistribution $W'$
enthalten die gleichen Informationen. Beide sind vierdimensional. $W'$
ist eine Quasidichteverteilung. Quasidichteverteilungen liefern zwar
den Erwartungswert aber die zugrundeliegenden Gr\"osse ist nicht wie
bei Wahrscheinlichkeitsverteilung unabh\"angig.

Wahrscheinlichkeitsaxiome:
\begin{align}
  P(E)&\in\mathbb{R},\quad P(E)\ge 0 \forall E \in F\\
  P(\Omega) &= 1\\
  P(E_1\cup E_2\cup\ldots) &= \sum_{i=0}^\infty P(E_i)
\end{align}
Ereignisraum $F$, Testraum $\Omega$


Die Wignerverteilung beschreibt die Amplitude eines Strahls an der
Position $x, y$ mit Richtung $p, q$. Da $\Gamma$ hermitisch ist, ist
$W$ stets reell. Destruktive Interferenz kann zu negativen Werten von
$W$ f\"uhren.

Das Integral von $W'$ \"uber die Raumkoordinaten $\r$ liefert ein
Winkelspektrum.

\begin{align}
  I(\p) = \int W'(\r,\p) \textrm{d}\r
\end{align}

Das Integral \"uber alle Winkelrichtungen $p$ liefert
die \"ortlich aufgel\"oste Intensit\"atsverteilung.

\begin{align}
  I(\r) = \frac{1}{(2\pi)^2} \int W'(\r,\p) \textrm{d}\p
\end{align}

H\"ohere Momente beschreiben beispielsweise die Qualit\"at von
Laserstrahlen.

Die Wignerverteilung $W'$ kann durch Faltung mit einem Gauss in eine
echte Dichteverteilung $Q$ verwandelt werden.

\begin{align}
  Q(\r,\p) = \int\int W'(\r,\p)\, e^{-\left(\frac{\r-\r'}{a}\right)^2} e^{-\left(\frac{\p-\p'}{b}\right)^2}\textrm{d}\r\textrm{d}\p
\end{align}

Insbesondere liefert eine Messung immer eine gegl\"attete
Wignerverteilung $Q$.


Im besten Fall: 

\begin{align}
  Q(\r,\p) = \frac{1}{\sqrt{\Delta x \Delta p }}\int\int W'(\r,\p)\, e^{-2 \left(\frac{\r-\r'}{\Delta x}\right)^2} e^{- 2 \left(\frac{\p-\p'}{\Delta p}\right)^2}\textrm{d}\r\textrm{d}\p
\end{align}

\begin{align}
  \Delta x \cdot \Delta p = \frac{\lambda}{\pi}
\end{align}

Das elektromagnetische Feld am Ausgang einer Multimode-Faser mit
monochromatischer Beleuchtung l\"asst sich folgendermassen in eine
Wignerverteilung umwandeln:

\begin{align}
  W'(x,p) = \int U(x+\Delta x/2) U^*(x-\Delta x/2) e^{-i k_0 \Delta x p}\textrm{d}\Delta x
\end{align}

Aus gegebener Wignerverteilung l\"asst sich das Feld jedoch nicht
rekonstruieren, denn im Ausdruck
\begin{align}
  U(x) = \frac{1}{\lambda U^*(0)} \int W'(x/2,p)\ e^{ik_0 px} \textrm{p}
\end{align}
taucht die Unbekannte $U(0)$, also die globale Phase auf.

Ein koh\"arenter Strahl kann beschrieben werden durch seine
Amplituden- und Phasenverteilung:

\begin{align}
  \label{eq:feld}
  U(x) = A(x) e^{i\Phi(x)}
\end{align}

Wobei die Phase $\Phi$ mit der Wellenrichtung $p$ im Zusammenhang steht:
\begin{align}
  p(x) = \frac{\lambda}{2\pi} \frac{\textrm{d} \Phi(x)}{\textrm{d} x}
\end{align}

Damit ist die Wignerverteilung f\"ur einen koh\"arenten Strahl
\begin{align}
  W'(x,p)= A^2(x) \cdot \delta\left(p-\frac{\lambda}{2\pi}
    \frac{\textrm{d} \Phi(x)}{\textrm{d} x}\right)
\end{align}

Ausgehend von einer holographisch gemessenen Feldverteilung
\eqref{eq:feld} l\"asst sich die Ableitung der Phase folgenderma\ss
en bestimmen:

\begin{align}
  \frac{\trmd U}{\trmd x} = A'(x) e^{i\Phi(x)} + A(x) e^{i\Phi(x)} i \Phi'(x)
\end{align}

\begin{align}
  \Phi'(x) = \textrm{Im}\left(\frac{\frac{\trmd U}{\trmd x}}{U} -  \frac{1}{| U |}\frac{\trmd |U|}{\trmd x} \right) 
\end{align}

Die Gr\"o\ss en $U$, $\trmd U / \trmd x$, $|U|$ and $\trmd |U| / \trmd
x$ k\"onnen leicht aus den gemessenen Daten ermittelt
werden. Insbesonder weist das Ergebnis $\Phi'(x)$ keine Probleme mit
Spr\"ungen um $2\pi$ auf, wie sie auftreten w\"urden, wenn einfach
$\trmd \arg(U)/\trmd x$ berechnet werden w\"urde.


Die \"Uberlagerung von zwei Feldern f\"uhrt zu einem Interferenzterm
$W'_\textrm{int}$

\begin{align}
  W'_\textrm{sum}(x,p) = W'_1(x,p) + W'_2(x,p) + W'_\textrm{int} (x,p)
\end{align}

mit 
\begin{align}
  W'_\textrm{int} (x,p) = 
  &\int U_1(x+\Delta x/2)  U^*_2(x-\Delta x/2) e^{i k_0 \Delta x p} \textrm{d} \Delta x \\
  + &\int U_2(x+\Delta x/2)  U^*_1(x-\Delta x/2) e^{i k_0 \Delta x p} \textrm{d} \Delta x
\end{align}




\end{document}